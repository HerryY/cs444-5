\documentclass[letterpaper,10pt,titlepage]{article}

\usepackage{graphicx}                                        
\usepackage{amssymb}                                         
\usepackage{amsmath}                                         
\usepackage{amsthm}                                          

\usepackage{alltt}                                           
\usepackage{float}
\usepackage{color}
\usepackage{url}

\usepackage{balance}
\usepackage[TABBOTCAP, tight]{subfigure}
\usepackage{enumitem}
\usepackage{pstricks, pst-node}

\usepackage{geometry}
\geometry{textheight=8.5in, textwidth=6in}

%random comment

\newcommand{\cred}[1]{{\color{red}#1}}
\newcommand{\cblue}[1]{{\color{blue}#1}}

\usepackage{hyperref}
\usepackage{geometry}

\def\name{Taylor Fahlman}

%% The following metadata will show up in the PDF properties
\hypersetup{
  colorlinks = true,
  urlcolor = black,
  pdfauthor = {\name},
  pdfkeywords = {cs444 ``operating systemsII''},
  pdftitle = {CS 444 Project 1},
  pdfsubject = {CS 444 Project 1},
  pdfpagemode = UseNone
}

\begin{document}
\include(gitlog.tex)

\section{COMMANDS}
Log of commands:\\
    \begin{itemize}
        \item mkdir /scratch/fall2015/cs444-24; cd /scratch/fall2015/cs444-24\\
        \item git clone git://git.yoctoproject.org/linux-yocto-3.14\\
        \item cp /scratch/opt/environment-setup-i586-poky-linux .\\
        \item cd linux-yocto-3.14; source ../environment-setup-i586-poky-linux\\
        \item cp /scratch/spring2015/files/config-3.14.26-yocto-qemu ./.config\\
        \item make -j4\\
        \item qemu-system-i386 -gdb tcp::5524 -S -nographic -kernel arch/i386/boot/bzImage -drive file=core-image-lsb-sdk-qemux86.ext3,if=virtio -enable-kvm -net none -usb -localtime --no-reboot --append "root=/dev/vda rw console=ttyS0 debug".\\
        \item \$GDB
        \item target target remote :5524
    \end{itemize}

\section{REFLECTIONS}
    1. I think the point of the assignment was to refresh us on using pthreads and to get us thinking in concurrency.\\
    2. I approached the problem piece by piece. First, how to create a thread, then buffer implementation, proudcer implementation, consumer implementation, random number generation, then handling of the full and empty buffers. This piece by piece approach made the algoriths/constructs fall into place.\\
    3. I tested by running the program as normal, but also testing the extremes. I filled the buffer with items and then tried to add onother to test the blocking when the buffer is full, and did the opposite to test when the buffer is empty. I also tested on different platforms to make sure the random number generation was handled correctly. \\
    4. I (re)learned about pthreads. I also leared about including asm inline, and the basics of concurrecy programming.\\ 

\begin{tabular}{l l l}\textbf{Detail} & \textbf{Author} & \textbf{Description}\\\hline
\href{git@github.com:fahlmant/cs444/commit/6da20559bf2fd5648a71e263be1fcddb68cb1786}{6da2055} & Taylor Fahlman & Added patch file\\\hline
\href{git@github.com:fahlmant/cs444/commit/7e4a6fad9eede1051a7ebc5db14bd3f360077816}{7e4a6fa} & Taylor Fahlman & reset file to semi-working copy\\\hline
\href{git@github.com:fahlmant/cs444/commit/886806cc8700041836984270d8457cb305283982}{886806c} & Taylor & One more cleanup\\\hline
\href{git@github.com:fahlmant/cs444/commit/d1c8f5c0597b0374c8fa150baf8654161f50eb83}{d1c8f5c} & Taylor & Updated writeup, cleaned up other files\\\hline
\href{git@github.com:fahlmant/cs444/commit/337f55d1a7a77b5840ab7cd55d3b98e6f8a4ea61}{337f55d} & Taylor Fahlman & Added writeup file\\\hline
\href{git@github.com:fahlmant/cs444/commit/16dfafe2b2ddaa55c00dc6f13d1624f5f348abdc}{16dfafe} & Taylor Fahlman & Added debugging statements\\\hline
\href{git@github.com:fahlmant/cs444/commit/4e313b07a7ce52c2eeb15e77780b0344fd58e776}{4e313b0} & Taylor Fahlman & Fixed init to match closer to noop, added printk for debugging\\\hline
\href{git@github.com:fahlmant/cs444/commit/48b1e091f6f8445fc355268ded9d2ab085f38217}{48b1e09} & Taylor Fahlman & Initialized the queue going forward:\\\hline
\href{git@github.com:fahlmant/cs444/commit/ee5d1be765a4359e795c1cfab79f45e7157dc427}{ee5d1be} & Taylor Fahlman & Fixed make errors\\\hline
\href{git@github.com:fahlmant/cs444/commit/19b5eed5fce8f0988d6de8e32433e71ec722f401}{19b5eed} & Taylor Fahlman & Fixed prink\\\hline
\href{git@github.com:fahlmant/cs444/commit/1b8337adbfd556c19331c13e47fdf30f6be4aa2c}{1b8337a} & Taylor Fahlman & Add sorter to put request in correct place and add prink statement\\\hline
\href{git@github.com:fahlmant/cs444/commit/0a48e33e7e336bcfa08225cb01f3df0fcaaed1f8}{0a48e33} & Taylor Fahlman & Add sorter to put request in correct place\\\hline
\href{git@github.com:fahlmant/cs444/commit/af1472c36acd59ba67588633806bf0893cfe8b8b}{af1472c} & Taylor Fahlman & Set request sector position\\\hline
\href{git@github.com:fahlmant/cs444/commit/4e4e0d50bb79cb0287dd9e15bf858950aca4e52c}{4e4e0d5} & Taylor Fahlman & Added prev and next request definitions\\\hline
\href{git@github.com:fahlmant/cs444/commit/264706d970dd72869b019450b1f75bc089303f51}{264706d} & Taylor Fahlman & Handled add request if the list was empty\\\hline
\href{git@github.com:fahlmant/cs444/commit/96e8b1364367fb70321afbf63863220dec0966bb}{96e8b13} & Taylor Fahlman & Added variables to add request\\\hline
\href{git@github.com:fahlmant/cs444/commit/09cf9a81142f457cc824439568eb4f7d0202b485}{09cf9a8} & Taylor Fahlman & Actually process request\\\hline
\href{git@github.com:fahlmant/cs444/commit/8fa13421d3f213f3e1698235bd37426b314838d6}{8fa1342} & Taylor Fahlman & Added print statement for debugging/proff\\\hline
\href{git@github.com:fahlmant/cs444/commit/2821d6ce67d5d1a5051838079defe48c4f14ee9f}{2821d6c} & Taylor Fahlman & Fixed makefile errors\\\hline
\href{git@github.com:fahlmant/cs444/commit/e9341bad2ae0d815c1eddedcb0070bb99c3ae231}{e9341ba} & Taylor Fahlman & Implemented backwards\\\hline
\href{git@github.com:fahlmant/cs444/commit/fc70a2c72e4933dbf271bf6f5b4adc250cd93f0b}{fc70a2c} & Taylor Fahlman & Handled going forward in requests\\\hline
\href{git@github.com:fahlmant/cs444/commit/95ceb4f9438107951331d241099b020416a2e580}{95ceb4f} & Taylor Fahlman & Checked if previous and next request are the same\\\hline
\href{git@github.com:fahlmant/cs444/commit/eea31387ddd4b439198607910a6025a957d4ce2b}{eea3138} & Taylor Fahlman & Commented to add notes\\\hline
\href{git@github.com:fahlmant/cs444/commit/a2427eaa94cfc72c3c636d359e1ff18729c50803}{a2427ea} & Taylor Fahlman & If list is not empty, create next and previous request pointers\\\hline
\href{git@github.com:fahlmant/cs444/commit/179f677a63ce6e59233d66e1696cc2a2c6da4f94}{179f677} & Taylor Fahlman & Added exit queue\\\hline
\href{git@github.com:fahlmant/cs444/commit/ef0f078626f6b3779db8239cfadefc9b0eb57422}{ef0f078} & Taylor Fahlman & Hopefully fixed strange differences between local and master\\\hline
\href{git@github.com:fahlmant/cs444/commit/e465e8bca4a176e46eec39c9506fcbbeaa52346d}{e465e8b} & Taylor Fahlman & Fixed make errors\\\hline
\href{git@github.com:fahlmant/cs444/commit/90fe379dcec284aec20892a7b5dfed7418d1f629}{90fe379} & Taylor Fahlman & Corrected typo fixes and makefile errors\\\hline
\href{git@github.com:fahlmant/cs444/commit/474cbfd6cf240364ec4739b91d21f0a08efa0507}{474cbfd} & Taylor Fahlman & Added struct and shell functions\\\hline
\href{git@github.com:fahlmant/cs444/commit/9a276f27141dec3a008c5ff9b3f53302579d55dc}{9a276f2} & Taylor Fahlman & added target to makefile\\\hline
\href{git@github.com:fahlmant/cs444/commit/41cc4baab6306e573e7cebb13ab23eba9ee28942}{41cc4ba} & Taylor Fahlman & Okay, really fixed formatting\\\hline
\href{git@github.com:fahlmant/cs444/commit/f744814a5f73441d92f08e7ec4c1734b8effe233}{f744814} & Taylor Fahlman & fixed formattin?\\\hline
\href{git@github.com:fahlmant/cs444/commit/3f40ac2f576b2250e924524e3c7973a7b57b99dc}{3f40ac2} & Taylor Fahlman & added look to default selection\\\hline
\href{git@github.com:fahlmant/cs444/commit/b49de190dda284a50bb2738f40b8436171b9d691}{b49de19} & Taylor Fahlman & Added inital sstf-iosched file and module init/exit\\\hline
\href{git@github.com:fahlmant/cs444/commit/1fd55679562034d28fcfbb6e2ac1642c5c88dfc9}{1fd5567} & Taylor Fahlman & Added LOOK as an option in the I/O scheduling config file\\\hline
\href{git@github.com:fahlmant/cs444/commit/d9c6b45e2f1eee4649ad42b115e750bd68d2622a}{d9c6b45} & Taylor Fahlman & Updated make clean\\\hline
\href{git@github.com:fahlmant/cs444/commit/f93baa8e24785744b1f3ae51ec64e45e62e6ea6e}{f93baa8} & Taylor Fahlman & Removed all pthread conditions\\\hline
\href{git@github.com:fahlmant/cs444/commit/2021542aebdf6ed1edee48fec117909c7b9926da}{2021542} & Taylor Fahlman & Changed control construct to while instead of if\\\hline
\href{git@github.com:fahlmant/cs444/commit/3a97f9ddf3e62cf13f868b6554fa48b76e200e64}{3a97f9d} & Taylor Fahlman & Added pritnf statements\\\hline
\href{git@github.com:fahlmant/cs444/commit/6ac82c37f00e5ac3d70cc9fce0cdd69a4b09a1c0}{6ac82c3} & Taylor Fahlman & Added print statements for plato and locke\\\hline
\href{git@github.com:fahlmant/cs444/commit/55e26ff34cc7fc3159c10da1add5024754667dbd}{55e26ff} & Taylor Fahlman & Added marx thread\\\hline
\href{git@github.com:fahlmant/cs444/commit/88de25a62a2e699650bc207419b2b384abc5a748}{88de25a} & Taylor Fahlman & Removed swp file\\\hline
\href{git@github.com:fahlmant/cs444/commit/ec8a5fbea281d6bffd2f3df386889152fadb13bc}{ec8a5fb} & Taylor Fahlman & Added marx function\\\hline
\href{git@github.com:fahlmant/cs444/commit/ba1a2150d205c820597bf84f89c025edfb785238}{ba1a215} & Taylor Fahlman & Added socrates thread\\\hline
\href{git@github.com:fahlmant/cs444/commit/edb136fac1c4ed5c5317f7a7a967bad090b32c86}{edb136f} & Taylor Fahlman & Added socrates function\\\hline
\href{git@github.com:fahlmant/cs444/commit/3cf1db8bc4cada1b6be913bc35423bd7d5f15855}{3cf1db8} & Taylor Fahlman & Added pythag thread\\\hline
\href{git@github.com:fahlmant/cs444/commit/d9a65baf438b66b6e0a9fd4cbfdf14bbb4244d5c}{d9a65ba} & Taylor Fahlman & Added pythagorus function\\\hline
\href{git@github.com:fahlmant/cs444/commit/8a6cceb7c92a712f5428de6f30d8c77f80e8dc16}{8a6cceb} & Taylor Fahlman & Added locke thread\\\hline
\href{git@github.com:fahlmant/cs444/commit/58ca58d458e413a75a8c02487bb65cf6acef3d59}{58ca58d} & Taylor Fahlman & Filled locke function\\\hline
\href{git@github.com:fahlmant/cs444/commit/ebf0d4a0840a6a6bb8b17916dc160c469197593d}{ebf0d4a} & Taylor Fahlman & Cleaned up after signal catch\\\hline
\href{git@github.com:fahlmant/cs444/commit/b48fefd146280d909c6cf2155b3fe1fdec0a3974}{b48fefd} & Taylor Fahlman & Added mt19937.h\\\hline
\href{git@github.com:fahlmant/cs444/commit/5e17f747c9f116ab75a330d8ce0bfdb3f3ec35ab}{5e17f74} & Taylor Fahlman & Added random num generator, eat, and think functions\\\hline
\href{git@github.com:fahlmant/cs444/commit/3426649ae1918162e2ffc19e8b76ffebcbfeafb6}{3426649} & Taylor Fahlman & Added infinite for loop\\\hline
\href{git@github.com:fahlmant/cs444/commit/f76dbb79c66c9c07cf4077e12042e9a2e0fd820f}{f76dbb7} & Taylor Fahlman & Added function pointer for proper pthread creation\\\hline
\href{git@github.com:fahlmant/cs444/commit/51d58b7179ad83063834c6bf5eac684dbe2dcea1}{51d58b7} & Taylor Fahlman & Added plato thread\\\hline
\href{git@github.com:fahlmant/cs444/commit/ffa8a6dcce81ffcaa6b0953ac7e66c13a5eebf23}{ffa8a6d} & Taylor Fahlman & Added pthread condition signals\\\hline
\href{git@github.com:fahlmant/cs444/commit/0b270f9d732ab6a652aaba7f6947f13c0e9cdc46}{0b270f9} & Taylor Fahlman & Added flag checking for forks\\\hline
\href{git@github.com:fahlmant/cs444/commit/b2297cd2ec70c735e39a1cc26d5ff39478f011bf}{b2297cd} & Taylor Fahlman & Restructed to use infinite loop\\\hline
\href{git@github.com:fahlmant/cs444/commit/0773ca18437eb1aff2874e73ff18fe67681602dc}{0773ca1} & Taylor Fahlman & Changed mutex pairs to int flags, and initiliazed mutexs\\\hline
\href{git@github.com:fahlmant/cs444/commit/1f5b835fc8b41d44ee33a2f2e6f5ec916591957a}{1f5b835} & Taylor Fahlman & Added mutex locks/unlocks to plato\\\hline
\href{git@github.com:fahlmant/cs444/commit/5d024e0d8a49d0db3b239929402e049376df57e1}{5d024e0} & Taylor Fahlman & added eat and think functions\\\hline
\href{git@github.com:fahlmant/cs444/commit/4795134ae4187793f395617ba52841d7c9e7f6d1}{4795134} & Taylor Fahlman & Labled philosophers\\\hline
\href{git@github.com:fahlmant/cs444/commit/d43a10f1ac0ff2b53122ca42969b95d5e5b474b9}{d43a10f} & Taylor Fahlman & added mutexes\\\hline
\href{git@github.com:fahlmant/cs444/commit/13edff973f75105ee082af2453bb0ea1789c6823}{13edff9} & Taylor Fahlman & Fixed merge conflicts\\\hline
\href{git@github.com:fahlmant/cs444/commit/084ab1563492ad141bbbf835df613426e6070df2}{084ab15} & Taylor Fahlman & Moved assignment 1 to subdir, added assignment2\\\hline
\href{git@github.com:fahlmant/cs444/commit/e9a06375c5e5ee5232d5a3a71fa1dd1ac336b7cc}{e9a0637} & Taylor & Finished writeup tex and pdf\\\hline
\href{git@github.com:fahlmant/cs444/commit/27eb37349447092eec979e1d09531408f38ac661}{27eb373} & Taylor Fahlman & Added bash file to transform gitlog to latex. Added in makefile targets\\\hline
\href{git@github.com:fahlmant/cs444/commit/2d454270c6b79091f4c1363da7185e3ffbcb2301}{2d45427} & Taylor Fahlman & Added writeup skeleton\\\hline
\href{git@github.com:fahlmant/cs444/commit/629da73d1a287be9db4919d9bc27c987acfb7128}{629da73} & Taylor Fahlman & Merge branch 'master' of https://github.com/fahlmant/cs444\\\hline
\href{git@github.com:fahlmant/cs444/commit/c96174c136e9194fd83f4d5fee5927432468fdfd}{c96174c} & Taylor Fahlman & Fixed tabbing\\\hline
\href{git@github.com:fahlmant/cs444/commit/6a7edbe421fb5e53a9a847a09117500c1ca1482e}{6a7edbe} & Taylor & Fixed blocking on full buffer\\\hline
\href{git@github.com:fahlmant/cs444/commit/007816390983ac7d562e8f9d4d31ff6abc7c40c8}{0078163} & Taylor & Removed producer sleep time because I couldn't get it to work\\\hline
\href{git@github.com:fahlmant/cs444/commit/9fb3e850a20f58cd174a80bf580c150ebb8a8860}{9fb3e85} & Taylor Fahlman & Added mersene twister files\\\hline
\href{git@github.com:fahlmant/cs444/commit/8e614a90d9b3adfedea1214253801c36f17c6aea}{8e614a9} & Taylor Fahlman & Added mersene twister generation. Added more cleanup in the signal catch. Moved pthread create to infinite loops so that it would get more numbers from the buffer\\\hline
\href{git@github.com:fahlmant/cs444/commit/6bc6226e01a9a1302c741331cce8ab8256f399ce}{6bc6226} & Taylor Fahlman & Added blocking on full or empty buffer. Added calls to random number generation.\\\hline
\href{git@github.com:fahlmant/cs444/commit/4bcd392bf2941eb49cc82229be9495cd4c9f62ee}{4bcd392} & Taylor Fahlman & Added pthread condition variables, and fixed rdrand macro\\\hline
\href{git@github.com:fahlmant/cs444/commit/f927d9b39907620e959c235c9b56ab0160bbc366}{f927d9b} & Taylor Fahlman & Fixed prototype of number generator function\\\hline
\href{git@github.com:fahlmant/cs444/commit/056005ccb725e1ce91c211bb5cc71acdd3cfe5e4}{056005c} & Taylor Fahlman & Fixed cpuid function\\\hline
\href{git@github.com:fahlmant/cs444/commit/97b39ebde5ea4090c9371d7868ec9984c6230703}{97b39eb} & Taylor Fahlman & Corrected CPUID define to differ from other global variables. Added rdrand generate define\\\hline
\href{git@github.com:fahlmant/cs444/commit/12d413cdbc773bea9797fbc8ffc523df368cae5c}{12d413c} & Taylor Fahlman & Added CPUID definition and fixed compiler errors and warning about function definitions\\\hline
\href{git@github.com:fahlmant/cs444/commit/a61e6320173be1ea668966396bb45bc7930d34f9}{a61e632} & Taylor Fahlman & Removed return that will never be executed\\\hline
\href{git@github.com:fahlmant/cs444/commit/4d08be28f8900f197846be9ac81d7c0d4c45cc03}{4d08be2} & Taylor Fahlman & Moved function defintions to correct place\\\hline
\href{git@github.com:fahlmant/cs444/commit/1a04be2d025bdf2f264d2fdcd2dad95ded4fb324}{1a04be2} & Taylor & Added random number generation function\\\hline
\href{git@github.com:fahlmant/cs444/commit/95628a0549663a27c0a3ef1f04118db28e89acda}{95628a0} & Taylor & Added signal.h so signals can work\\\hline
\href{git@github.com:fahlmant/cs444/commit/a4f7c992ec397d9b510126246137537b77e55930}{a4f7c99} & Taylor & typo fixes\\\hline
\href{git@github.com:fahlmant/cs444/commit/29e0fd1c29a38fec3bd5a999d8bf8efa513c70fd}{29e0fd1} & Taylor & Added signal catch and reimplemented buffer as queue\\\hline
\href{git@github.com:fahlmant/cs444/commit/2274e684608cf836dc48bda6d3ec5d5d9b66f5fb}{2274e68} & Taylor Fahlman & Added mutex lock/unlock to consumer\\\hline
\href{git@github.com:fahlmant/cs444/commit/7249be01ba7edbda6723b5acbe59f46efa2886a9}{7249be0} & Taylor Fahlman & Fixed issues with printing to stdout\\\hline
\href{git@github.com:fahlmant/cs444/commit/050904eeb0342d93e297fec347a08544cc99ce07}{050904e} & Taylor Fahlman & Added mutex init, and locking/releasing on produce\\\hline
\href{git@github.com:fahlmant/cs444/commit/50cf97754d7e3aa081ef30f56af307d2671969ff}{50cf977} & Taylor Fahlman & Changed global buffer pointer to a variable. Changed reference to it so that it could be written to and read from\\\hline
\href{git@github.com:fahlmant/cs444/commit/c5c13a973ba6f8363baf95b9e2330ea4dca16651}{c5c13a9} & Taylor Fahlman & Changed pthread create functions to not segfault\\\hline
\href{git@github.com:fahlmant/cs444/commit/7ff17bdada1706eb1057ba678bde47c8f5b18774}{7ff17bd} & Taylor Fahlman & Outlined needed steps in produce\\\hline
\href{git@github.com:fahlmant/cs444/commit/d09e23095844c4b70beb8f084938a6716eb11f24}{d09e230} & Taylor Fahlman & Changed the global buffer to a struct\\\hline
\href{git@github.com:fahlmant/cs444/commit/209f7513f40020a3ae61565dd7f115b7c72b8472}{209f751} & Taylor Fahlman & Added comments for placeholders and fixed print statement\\\hline
\href{git@github.com:fahlmant/cs444/commit/5d9edbd4229b7263d39225f1ec581258af623a3d}{5d9edbd} & Taylor Fahlman & Removed unneeded semaphore .h>\\\hline
\href{git@github.com:fahlmant/cs444/commit/2c5734f5d2cbbf0650eefb5acbf3da30c458f406}{2c5734f} & Taylor Fahlman & Updated assignment 1\\\hline
\href{git@github.com:fahlmant/cs444/commit/4b56781b7d05dd3c5653e92d8b3bfe43374a4190}{4b56781} & Taylor Fahlman & Changed the buffer to be a pointer to an array of 32 buffer items\\\hline
\href{git@github.com:fahlmant/cs444/commit/6445be50d9acb170ec3adaa80fe9418b160faf23}{6445be5} & Taylor Fahlman & Added sleep and print in consume function\\\hline
\href{git@github.com:fahlmant/cs444/commit/f4facc9d5a16eb42168ac97078b0f487a08a407c}{f4facc9} & Taylor Fahlman & Fixed compile issues\\\hline
\href{git@github.com:fahlmant/cs444/commit/607fa54aeb02833e23feaaebe3ca42504e86a20c}{607fa54} & Taylor Fahlman & Filled consume function\\\hline
\href{git@github.com:fahlmant/cs444/commit/2244c720455b015276a94f97610bd3e710e41705}{2244c72} & Taylor Fahlman & Assigned buffer pointer correctly to the address of the item with dummy values\\\hline
\href{git@github.com:fahlmant/cs444/commit/4d9e3d84c0309caf5af3ca5aed2af08010b2ef85}{4d9e3d8} & Taylor Fahlman & Filled general outline of produce function\\\hline
\href{git@github.com:fahlmant/cs444/commit/e2d0eedfaec7397cf59d7f1fd46886026f0f1727}{e2d0eed} & Taylor Fahlman & Makefile and assignment1.c update\\\hline
\href{git@github.com:fahlmant/cs444/commit/d4c0a889450f5e33eeee262f56cea203a29aed0a}{d4c0a88} & Taylor Fahlman & Makefile:     Added make file will the most common flags for style and debugging. Used a two step compile process, .o then an executable\\\hline
\href{git@github.com:fahlmant/cs444/commit/15a4c12977816ab8bab1fe5f5a9b3c3774a8367a}{15a4c12} & Your Name & Added concurrency\\\hline
\href{git@github.com:fahlmant/cs444/commit/1ca41740bd9248799f23d5798a4a82d99e63ba5a}{1ca4174} & Your Name & Added linux kernel\\\hline\end{tabular}

%input the pygmentized output of mt19937ar.c, using a (hopefully) unique name
%this file only exists at compile time. Feel free to change that.
\end{document}
